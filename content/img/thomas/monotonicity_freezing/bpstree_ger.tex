\documentclass[crop,tikz]{standalone}
\usetikzlibrary{%
    arrows,
    arrows.meta,
    backgrounds,
    calc,
    decorations.pathreplacing,
    fit,
    matrix,
    positioning,
    scopes,
    shadows
}
\usepackage[linguistics]{forest}
\usepackage[charter]{mathdesign}
\tikzset{headarrow/.style = {-{Latex[length=.5em]}}}
\tikzset{move/.style = {dashed,blue,headarrow}}

\newcommand{\mlex}[2]{\ensuremath{\textrm{#1} ::\thinspace \mathrm{#2}}}
\newcommand{\fsel}[1]{\ensuremath{\mathrm{#1^+}}}
\newcommand{\fcat}[1]{\ensuremath{\mathrm{#1^-}}}
\newcommand{\flcr}[1]{\ensuremath{\mathrm{#1^+}}}
\newcommand{\flce}[1]{\ensuremath{\mathrm{#1^-}}}
\newcommand{\fadj}[1]{\ensuremath{\mathrm{#1^\sim}}}
\newcommand{\Merge}{Merge}
\newcommand{\Move}{Move}
\newcommand{\Adjoin}{Adjoin}

\begin{document}
\begin{forest}
    [CP
        [VP, name=target-VP
            [$t$, name=source-obj]
            [gelesen]
        ]
        [C$'$
            [hat]
            [TP
                [DP, name=target-obj
                    [das]
                    [Buch]
                ]
                [TP
                    [DP, name=target-nom
                        [der]
                        [Hans]
                    ]
                    [T$'$
                        [T]
                        [$\mathit{v}$P
                            [$t$, name=source-nom]
                            [$t$, name=source-VP]
                        ]
                    ]
                ]
            ]
        ]
    ]
    %
    \draw[move] (source-nom) [out=160, in=335] to (target-nom);
    \draw[move] (source-VP) [out=220, in=185] to (target-VP);
    \draw[move] (source-obj) [out=290, in=195] to (target-obj);
\end{forest}
\end{document}
